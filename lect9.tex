\paragraph{Sentence} $a_n > 0$ monotonous non-increasing. Then $\sum a_n$ and $\sum 2^na_{2^n}$ converge/diverge together.
\subparagraph{Proof}
Denote partial sums as $A_n= \sum_{k=1}^n a_k$ and $S_n= \sum_{k=1}^n 2^ka_{2^k}$. Lets show $\left\{ A_n \right\}$ bounded $\iff \left\{ A_n \right\}$ bounded.
$$\forall k \: 2^ka^{2^{k+1}} \leq a^{2^k+1}+a^{2^k+2}+\dots+a^{2^{k+1}} \leq 2^ka^{2^k}$$
\begin{align*}
&\sum_{k=0}^{n-1} 2^ka^{2^{k+1}} &\leq \sum_{k=2}^{2^n} a_n &\leq \sum_{k=0}^{n-1} 2^ka^{2^k}&\\
&\frac{1}{2}\left(S_{n+1}-a_1\right) &\leq A_{2^n}-a_1 &\leq S_n&\\
\end{align*}
$\left\{ A_n \right\}$ bounded $\iff \left\{ A_n \right\}$ bounded.
\paragraph{Example}
$$\sum_{n=1}^{infty} \frac{1}{n^p}\quad p\in\mathbb{R}$$
For $p\leq 0$ diverges. Assume $p>0$.
$$a_n = \frac{1}{n^p}$$
$$2^na_{2^n} = 2^n \frac{1}{{2^n}^p} = \frac{1}{2^{(p-1)n}}$$
This is geometrical series, it converges if $q<0 \Rightarrow 2^{1-p} < 1 \Rightarrow p > 1$.
\paragraph{Integral test}$f\geq0$, non-increasing on $[0,\infty)$ and integrable $\forall c> 0$ on $[0,c]$. Denote $a_n = f(n)$, then $\sum_{n=1}^\infty a_n$ and $\int_0^\infty f dx$ converge/diverge together.
\subparagraph{Proof}
Let's look on $[a,n]$ and partition $P$: $0<1<2<\dots<n$.
$$U(P,f) = f(0)+f(1)+\dots+f(n-1) = f(0)+S_{n-1}$$
$$L(P,f) = f(1)+f(2)\dots+f(n) = S_n$$
This means
$$f(0)+S_{n-1} \leq \int_0^n f dx \leq S_n  $$
\paragraph{Example}
$$\sum \frac{1}{n\ln n}$$
\subparagraph{Solution}
take a look on function $f(x) = \frac{1}{x\ln x}$:
$$\int_2^c \frac{dx}{x\ln x} = \int_{\ln 2}^{\ln c} \frac{1}{u} du = \left[ \ln u \right]_{\ln 2}^{\ln c} = \ln(\ln c) - \ln(\ln 2) \to \infty$$
Meaning series diverges.
\paragraph{Logarithmic test} For $$K = \lim_{n\to \infty} n \log \frac{a_{n}}{a_{n+1}}$$ series converge if $K<1$ and diverge if $K>1$.
\paragraph{Sentence} $a_n \geq$. If $\sum a_n$ converges, then $\sum b_n$ which is obtained by shuffling $a_n$ converges to same sum.
\subparagraph{Proof}
Denote $b_n = a_{\pi(n)}$, then $\pi(n)$ is bijective.

Also denote $S_n = \sum_{k=1}^n a_n$ and $T_n = \sum_{k=1}^n b_n$. Denote $m = \max \big\{ \pi(1),\pi(2), \pi(3), \dots, \pi(n) \big\}$, then $T_n \leq S_m \leq S \Rightarrow T_n \leq S$. Which means $b_n$ converges to some $T \leq S$. From symmetry, $S \leq T$, i.e. $T=S$.
\paragraph{Leibniz Test} Let $\forall n \: a_n > a_{n+1}$ and $a_n \to 0$, then $\sum_{n=1}^\infty \left(-1\right)^{n+1} a_n$ converges and $0 < S < a_n$ and for remained $R_k = S - S_k$  $|R_k| < a_{k+1}$.
\subparagraph{Proof}
$$S_{2n-1} = a_1 - (a_2-a_3) - (a_4-a_5) - \dots - (a_{2n-2}-a_{2n-1})$$
$$S_{2n+1} = S_{2n-1} - (a_{2n}-a_{2n+1}) \leq S_{2n-1}$$
This means $S_{2n-1}$ is decreasing sequence and $S_{2n-1} \leq a_1$.
At the same time
$$S_{2n} = (a_1-a_2)+(a_3-a_4)+\dots+(a_{2n-1}-a_{2n})$$
$$S_{2n+2} = S_{2n}+(a_{2n+1}-a_{2n+2}) \geq S_{2n}$$
This means $S_{2n}$ is increasing sequence. Since $S_{2n} =S_{2n-1}-a_{2n} < S_{2n-1}$, intervals $\left\{ \left[ S_{2n-1}, S_{2n} \right] \right\}$ are inside each other, then from Cantor's lemma $\bigcap \left[ S_{2n-1}, S_{2n} \right] = S$ and $S_{2n} \to S$ and $S_{2n-1} \to S$, i.e. $S_n \to S$ and also $0<S<a_1$ from monotonousness of two subsequences.

Moreover, $R_k = \sum_{n=k+1}^{\infty} (-1)^{n+1}a_n$ is also alternating series.
$$R_k = (-1)^{k+2}a_{k+1} + (-1)^{k+3}a_{k+2} + \dots = (-1)^{k+2}\left[a_{k+1} + (-1)^{1}a_{k+2} + \dots\right]$$
And thus $|R_k| < a_{k+1}$
\paragraph{Example}
$$\sum (-1)^n \frac{1}{n^p} \quad p>0$$
converges. Note that for $0 < p \leq 1$ series doesn't converges absolutely.
\paragraph{Sentence}
If exists $L = \lim_{n\to \infty} \left| \frac{a_{n+1}}{a_n} \right|$ or $K = \lim_{n\to \infty} \left| \sqrt[n]{a_n} \right|$ then
\begin{enumerate}
	\item If $L<1$ or $K<1$ then series converges
	\item If $L >1$ or $K>1$ then series diverges
\end{enumerate}
\subparagraph{Proof}
\begin{enumerate}
	\item Converges absolutely
	\item $|a_n| \not \to 0$ i.e. $a_n \not \to 0$
\end{enumerate}