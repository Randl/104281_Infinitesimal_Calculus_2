\section{Improper integrals}
Let $D$ set in $\mathbb{R}$ whose boundary is of area 0 and $f$ defined and continuous on $D$ except maybe set of area 0. (Neither $D$ nor $f$ must be bounded).

Lets try take
$$D_1 \subseteq D_2 \subseteq D_3 ...$$
$$D = \bigcup D_i$$
Define
$$\iint\limits_D f = \lim_{i \to \infty} \iint\limits_{D_i} f$$
\paragraph{Example}
$$f = \begin{cases}
-1&y>x\\1&y\leq x
\end{cases}$$
For two different $D_i$: $D_n = [0,n] \times [0,2n]$ and $D^\prime_n = [0,n]^2$:
$$\iint\limits_{D_n} f = -n^2 \to \infty$$ 
$$\iint\limits_{D^\prime_n} f = 0 \to 0$$
\paragraph{Example} 
$$f = \begin{cases}
\frac{1}{\sqrt{2}}(x+y)&y>x\\-x&y\leq x
\end{cases}$$
$$D_n = =[0,n]^2$$
$$D^\prime_n = \left\{ (x,y), x\geq 0, y \geq 0, x^2+y^2\leq n^2  \right\}$$
$$\iint_{D_n} = -\iint\limits_{L_n} +\frac{1}{\sqrt{2}}\iint\limits_{U_n}$$
$$I_L = \int_0^n \int_0^x x dy dx= \int_0^n \left[xy\right]_{y=0}^{y=x} dx = \int_0^n x^2 dx = \frac{n^3}{3}$$
$$I_U  = \frac{1}{\sqrt{2}} \int_0^n \int_x^n (x+y) dx = \frac{1}{\sqrt{2}} \int_0^n \left[ xy+\frac{y^2}{2} \right]_{y=x}^{y=n} dx = \frac{1}{\sqrt{2}} \int_0^n nx+\frac{n^2}{2}+\frac{3x^2}{2}dx = \frac{1}{\sqrt{2}} \left[ \frac{nx^2}{2}+\frac{n^2x}{2}+\frac{x^3}{2}\right]_0^n = \frac{1}{2\sqrt{2}}n^3$$

Lets show that $\iint\limits_{D_n^\prime} f = 0$. Or in other words
$$\iint\limits_{U_2} f= - \iint\limits_{L_2} f$$
$$\iint\limits_{U_2} \frac{1}{\sqrt{2}}(x+y)= \iint\limits_{L_2}x$$
Define $\phi: L_2 \to U_2$ as rotation by $\frac{\pi}{4}$
$$\phi\begin{pmatrix} x\\y \end{pmatrix} = \begin{pmatrix} \frac{1}{\sqrt{2}}&-\frac{1}{\sqrt{2}}\\\frac{1}{\sqrt{2}}&\frac{1}{\sqrt{2}} \end{pmatrix} \begin{pmatrix} x\\y \end{pmatrix} = \begin{pmatrix} \frac{1}{\sqrt{2}}(x-y)\\\frac{1}{\sqrt{2}}(x+y) \end{pmatrix} $$
$$\phi(s,t) = \left(\frac{1}{\sqrt{2}}(s-t),\frac{1}{\sqrt{2}}(s+t)\right)$$
$$\iint\limits{U_2} \frac{1}{\sqrt{2}} (x+y) = \iint\limits_{L_2} \frac{1}{\sqrt{2}}\left(\frac{1}{\sqrt{2}}(s-t)+\frac{1}{\sqrt{2}}(s+t)\right) = \iint\limits_{L_2} s$$
\paragraph{Definition}
Suppose $f\geq 0$.  Then if $D_1 \subseteq D_2 \subseteq D$
$$\iint\limits_{D_2} f \geq \iint\limits_{D_1} f$$
Define
$$\iint\limits_{D} f = \sup \left\{ \iint\limits_{B}, f_B, B \:- \:bounded \right\}$$
$f$ is integrable on $D$ if
$\iint\limits_{D} f$ is finite.
\paragraph{Sentence}  Under those conditions if $f$ is integrable and let $A_1 \subseteq A_2 \subseteq \dots \subseteq D$ closed bounded sets with boundary of area 0 such that $f_{A_n}$ bounded for all $n$ such that $\forall B \subseteq D \exists n \: B \subseteq A_n$ then
$$\iint\limits_D = \lim_{n \to \infty} \iint\limits_{A_n} f$$

And vice versa: if $A_n$ sequence as in previous paragraph, and exists finite
$$\lim_{n \to \infty} \iint\limits_{A_n} f$$
then $f$  is integrable and
$$\iint\limits_D = \lim_{n \to \infty} \iint\limits_{A_n} f$$
\subparagraph{Proof}
For such $A_n$
$$\iint\limits_{A_n} f \leq \iint\limits_{A_{n+1}} f $$
$$\lim_{n \to \infty} \iint\limits_{A_n} f$$
exists since it's monotonous sequence.
Suppose $f$ is integrable. Then $\forall \epsilon>0 \exists B \subseteq D $ such that $f_D$ is bounded and
$$\iint\limits_{B} f \geq \iint\limits_{D} - \epsilon$$
$$\exists N \: B \subseteq A_N \Rightarrow \iint\limits_{D}  - \epsilon \leq \iint\limits_{B} \leq \iint\limits_{A_N} \to \lim_{n \to \infty} \iint\limits_{A_N} f$$
$$\iint\limits_{D}  - \epsilon \leq \lim_{n \to \infty} \iint\limits_{A_N} f \leq \iint\limits_{D}  $$ 

For any closed and bounded $B$ such that $f_B$ is bounded $\exists n \: B \subseteq A_N$
$$\iint\limits_{B} f \subseteq \iint\limits_{A_N} f \leq \lim_{n \to \infty}\iint\limits_{A_N} f  = M $$
$$\iint\limits_D f \leq M$$
\paragraph{Example}
Calculate
$$\iint\limits_{\mathbb{R}^2} e^{-(x^2+y^2)}$$
Let's calculate with $A_n = B(0,n) = \left\{ (x,y): (x^2+y^2) \leq n^2 \right\}$
$$\iint\limits_{A_n} e^{-(x^2+y^2)} = \int_0^n \int_0^{2\pi} e^{-r^2}r d\theta dr = \int_0^n 2\pi r e^{-r^2} dr = \pi \int_0^{n^2} e^{-u}du = \pi \left[ -e^{-u} \right]_0^n = \pi (-e^{-n^2} + 1) \to \pi$$
Now lets take $D_n = [-n,n]^2$
$$\int_{-n}^{n}\int_{-n}^{n} e^{-(x^2+y^2)}dxdy = \left(\int_{-n}^{n} e^{-x^2}dx\right)^2 \to \pi$$
Then
$$\int_{-n}^{n} e^{-x^2}dx \to \sqrt{\pi}$$
$$\int_{-\infty}^{\infty} e^{-x^2}dx = \sqrt{\pi}$$
\paragraph{Non-positive $f$}
Define
$$f^+(x,y) = max(f(x,y), 0)=\begin{cases}
f(x,y)=|f(x,y)|& f(x,y) \geq 0\\0&f(x,y) \leq 0
\end{cases}$$
$$f^+(x,y) = -min(f(x,y), 0)=\begin{cases}
-f(x,y)=|f(x,y)|& f(x,y) \leq 0\\0&f(x,y) \geq 0
\end{cases}$$
Then
$f = f^+ - f^-$ and $|f| = f^+ + f^-$.
We say that $f$ is integrable on $D$ if $f^+$ and $f^-$ are integrable and
$$\iint\limits_{D} f= \iint\limits_{D} f^+ - \iint\limits_{D} f^-$$
That also means that $f$ is integrable $\iff$ $|f|$ is integrable.
\paragraph{Example}
$$\iint\limits_{x^2+y^2\leq 1} \frac{x^2}{(x^2+y^2)^{\frac{5}{2}}}$$
$$D_n = \left\{ (x,y): \frac{1}{n} \leq x^2+y^2 \leq 1 \right\}$$
$$\iint\limits_{D_n} \frac{x^2}{(x^2+y^2)^{\frac{5}{2}}} = \int_0^{2\pi}\int_{\frac{1}{n}}^1 \frac{r^2 \cos^2 \theta}{r^5}r dr d\theta =  \int_0^{2\pi}\int_{\frac{1}{n}}^1 \frac{\cos^2 \theta}{r^2} dr d\theta = \int_0^{2\pi}\cos^2 \theta d\theta \int_{\frac{1}{n}}^1 \frac{1}{r^2} dr \to \infty$$
\paragraph{Example}
Find area between two hyperbolas:
$y^2-x^2=1$ and $x^2-y^2=1$.
Lets divide area into 8 equal parts with axis and lines $y=\pm x$. 
$$A_n = D_0 \cap [0,n]^2$$
Then
$$\iint\limits_{A_n} 1 = \iint\limits_{A_n} 1 dx dy$$
Divide integral into two parts
$$\iint\limits_{A_n} 1 dx dy = \underbrace{\int_0^1 \int_{0}^{x} 1 dy dx}_{\frac{1}{2}} + \int_1^n \int_{\sqrt{x^2-1}}^x 1 dy dx $$
$$\int_1^n \int_{\sqrt{x^2-1}}^x 1 dy dx = \int_1^n x-\sqrt{x^2-1} dx$$
Is $A(D) < \infty$?

$A(D) < \infty$ $\iff$ $\int_1^n x-\sqrt{x^2-1} dx < \infty$.
$$\int_1^n x-\sqrt{x^2-1} dx = \frac{1}{x+\sqrt{x^2+1}} \geq \frac{1}{2x}$$
which doesn't exists.
