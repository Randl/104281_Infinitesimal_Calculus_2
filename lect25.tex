\paragraph{Example}
$$\iint\limits_{[0,1]^2} \sqrt{x^2+y^2} $$
Switch to polar coordinates:
$$x = r\cos \theta$$
$$y = r \sin \theta$$
We can calculate the integral only only lower half (triangle) of the square:
$$0 \leq \theta \leq \frac{\pi}{4}$$
$$0 \leq r \leq \frac{1}{\cos \theta}$$
\begin{align*}
\iint \sqrt{x^2+y^2} = \iint\limits_{D} r^2 = \int_0^{\frac{\pi}{4}}\int_0^{\frac{1}{\cos \theta}} r^2 dr d\theta = \int_0^{\frac{\pi}{4}} \left[ \frac{r^3}{3} \right]_0^{\frac{1}{\cos \theta}} d\theta = \int_0^{\frac{\pi}{4}} \frac{1}{3\cos^3 \theta} d\theta =\\= \frac{1}{3}\left[ \frac{1}{2}\frac{\tan \theta}{\cos \theta} \right]_{0}^{\frac{\pi}{4}}+ \frac{1}{2\cdot 3} \int_0^{\frac{\pi}{4}} \frac{1}{\cos \theta} d\theta = \frac{1}{6}\left[\frac{\tan \theta}{\cos \theta} +  \ln(\cos \frac{\theta}{2}-\sin \frac{\theta}{2})+\ln(\cos \frac{\theta}{2}+\sin \frac{\theta}{2})\right]_{0}^{\frac{\pi}{4}} = \dots
\end{align*}
\paragraph{Note} It's possible to use polar coordinates even if point $(0,0)$ is inside the interval, even though $\theta$ isn't defined there and jacobian is 0 there. Since for
$$R_\epsilon = \left\{ (x,y)\in R: x^2+y^2 \geq \epsilon_0 \right\}$$
$$\iint\limits_{R_\epsilon} f\to \iint\limits_{R} f$$
$$\left| \iint\limits_{R \setminus R_\epsilon} f\right| \leq MA(R \setminus R_\epsilon) \leq M\pi \epsilon^2 \to 0$$
\paragraph{Example}
$$\iint\limits_{R} \cos \frac{\pi(2x-y)}{2(x-2y)}$$
When $R$ is triangle of lines $y=x$, $y=-x$ and $y=\frac{1}{2}(x-2)$.
Since $y=\frac{1}{2}(x-2)$ is $x-2y=1$ lets choose
$$\begin{cases}
s=x-2y\\t=2x-y
\end{cases}$$
Or
$$\begin{cases}
x=\frac{2}{3}t-\frac{1}{3}s\\y=\frac{1}{3}t-\frac{2}{3}s
\end{cases}$$
$$J(\phi) = \det \begin{pmatrix}\frac{2}{3}&-\frac{1}{3}\\\frac{1}{3}&-\frac{2}{3}\end{pmatrix} = -\frac{1}{3}$$
$$|J(\phi)| = \frac{1}{3}$$
$$x-2y = 2 \rightarrow s=2$$
$$y=x \rightarrow \frac{2}{3}t-\frac{1}{3}s = \frac{1}{3}t-\frac{2}{3}s \rightarrow t=-s$$
$$y=-x \rightarrow t=s$$
$$I = \frac{1}{3}\iint\limits_{D} \cos \frac{\pi t}{2s} = \int_0^2 \int_{-s}^s \cos \frac{\pi t}{2s} dt ds = \frac{1}{3}\int_0^2 \left[ \frac{2s}{\pi}\sin \frac{\pi t}{2s} \right]_{t=-s}^{t=s}ds = \frac{1}{3}\int_0^2 \frac{2s}{\pi}\cdot 1 - \frac{2s}{\pi} \cdot (-1) ds = \frac{4}{3\pi}\int_0^2 s ds = \frac{1}{6\pi}$$
\paragraph{Note}
$$\left|J(\phi^{-1})\right| = \left| J(\phi)^{-1} \right|$$
\subparagraph{Proof}
$$\begin{cases}
x=x(s,t)\\y=y(s,t)
\end{cases}$$
$$\begin{cases}
s=s(x,y)\\t=t(x,y)
\end{cases}$$
$$\forall x,y \: x(s(x,y),t(x,y)) = x$$
Find derivative of both sides by $x$ and $y$
$$\frac{\partial x}{\partial s} \cdot \frac{\partial s}{\partial x} + \frac{\partial x}{\partial t} \cdot \frac{\partial t}{\partial x} = 1$$
$$\frac{\partial x}{\partial s} \cdot \frac{\partial s}{\partial y} + \frac{\partial x}{\partial t} \cdot \frac{\partial t}{\partial y} = 0$$
Same for the second equation.
$$\begin{pmatrix}\frac{\partial x}{\partial s}&\frac{\partial x}{\partial t}\\\frac{\partial y}{\partial s}&\frac{\partial y}{\partial t}
\end{pmatrix} \cdot \begin{pmatrix}\frac{\partial s}{\partial x}&\frac{\partial s}{\partial y}\\\frac{\partial t}{\partial x}&\frac{\partial t}{\partial y}
\end{pmatrix}=\begin{pmatrix}1&0\\0&1\end{pmatrix}$$
\paragraph{Area of interval}
$$A(D) = \iint\limits_D 1$$
$D$ is bounded by $x^2=by$, $x^2=ay$, $y^2=qx$, $y^2=px$, where $0<p<q$ and $0<a<b$.
$$t=\frac{x^2}{y}$$
$$s=\frac{y^2}{x}$$
$$|J(\phi)| = \left|\frac{1}{J(\phi^{-1})} \right| = \left[\det \begin{pmatrix}\frac{2x}{y}&-\frac{x^2}{y^2}\\-\frac{y^2}{x^2}&\frac{2y}{x}\end{pmatrix}\right]^{-1} = \frac{1}{4-1}=\frac{1}{4}$$
$$I=\frac{1}{3}\iint\limits_{D} 1 = \frac{1}{3}\int_a^b\int_p^q dsdt = \frac{1}{3}(b-a)(q-p)$$
\paragraph{Example}
Calculate area inside curve:
$$(x^2+y^2)^2=2(x^2-y^2)$$
Switch to polar:
$$r^4=(r^2\cos^2 \theta - r^2 \sin^2 \theta) = 2r^2\cos 2\theta \Rightarrow r^2 = 2\cos 2\theta$$
$$I = \iint\limits r dr d\theta = \int_0^{\frac{\pi}{4}} \int_{0}^{\sqrt{2\cos 2\theta}} dr d\theta$$