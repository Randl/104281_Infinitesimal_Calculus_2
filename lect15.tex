\paragraph{Elementwise derivative and integral}
Series with radius of convergence $R$ define function on this radius.
$$f(x) = \sum a_n x^n$$
We know that
\begin{itemize}
	\item $f$ is continuous
	\item Radius of convergence of $ \sum \frac{a_n}{n+1} x^{n+1}$
	is $R$ and $\forall r > R$ $f$ is integrable on $[-r,r]$ and integral is equal to 
	$$\int_0^x f(t) dt = \sum_{n=0}^{\infty} \frac{a_n}{n+1} x^{n+1}$$
	Convergence in points $R$ and $-R$ for integral is same as for series itself.
	\item Elementwise derivative: radius of convergence of $ \sum na_nx^{n-1}$ is R. $\forall r > R$  $f$ is derivable on $(-R,R)$ and derivative is equal
	$$f^\prime = \sum na_nx^{n-1}$$
	Convergence in points $R$ and $-R$  for  series is same as for series derivative (derivative is one-sided).
\end{itemize}
\subparagraph{Proof}
\begin{itemize}
	\item Let $R$ be a radius of convergence. We've seen that$\forall 0\leq r< R$ converges and convergence in $[-r,r]$ for any $r$, thus $f$ is continuous on $(-R,R)$.
	
	If series converge in $R$, then from previous sentence $f$ converges uniformly on $[0,R]$ $\Rightarrow$ $f$ is continuous on $[0,R]$ including endpoints. 
	\item $$R_i = \frac{1}{\limsup \sqrt[n]{\frac{|a_n|}{n+1}}} = \frac{1}{\limsup \sqrt[n]{|a_n|} \underbrace{\sqrt[n]{\frac{1}{n+1}}}_{1}} = \frac{1}{\limsup \sqrt[n]{|a_n|}} = R$$
	Rest of the statement is proofed by what's proofed for function series.
	\item Similarly for derivatives
\end{itemize}
\paragraph{}
It is possible to apply derivative more than once.
\paragraph{Abel sentence}
If $\sum a_n$ converges than $\forall 0 \leq x < 1$ $\sum a_n x^n$ absolutely converges.
$$\sum_{n=0}^\infty a_n x^n \stackrel{x\to 1^-}{\to} \sum a_n$$
\subparagraph{Proof}
Absolute convergence is from lemma we proofed. 
$\sum a_n$ converges $\Rightarrow$ radius of $a_n x^n$ greater or equal than 1. Convergence on $[0,1]$ is uniform, thus $\sum a_n x^n$ defines continuous function in 1 and result is straightforward from here.
\paragraph{Example}
$$\sum \frac{1}{n!}x^n$$
Radius of convergence $R=\infty$.
By deriving:
$$f^\prime(x) = \sum \frac{n}{n!}x^{n-1} = \sum \frac{1}{(n-1)!}x^{n-1}$$
Since function is equal to own derivative, $f(x)= ce^x$, in $x=0$ $c=1$, i.e. $f(x) = e^x$.
\paragraph{Example}
From previous example
$$e^x =\sum_{n=0}^\infty \frac{1}{n!}x^n$$
$$\forall t \: e^{-t^2} = \sum_{n=0}^\infty \frac{1}{n!}(-1)^n t^{2n}$$
with $R=\infty$.
By integration
$$\int_0^x e^{-t^2}dt = \sum_{n=0}^\infty \frac{(-1)^n}{n!} \frac{1}{2n+1} x^{2n+1}$$
\paragraph{}
Let $f$ function defined on $(-R,R)$. Is there power series equal to $f$.

Suppose there is. That means $f$ is derivable infinite number of times. Let
$$f(x) = \sum_{n=0}^\infty a_n x^n$$
$$f(0) = a_0$$
By derivation
$$f^\prime(0) = a_1$$
By repeating the process, we get
$$f^{(k)}(0) = k!a_k$$
That means that if there exists power series equal to function, than it's unique. Also $a_n = \frac{1}{n!}f^{(n)}(0) $.
If $f$ is equal to power series on $(-R,R)$ then
$$f(x) = \sum_{n=0}^\infty \frac{1}{n!}f^{(n)}(0)x^n$$
which is Taylor series of $f$. 
\paragraph{Note} Not every function can be written as power series (just like Taylor series). For example
$$f(x) = \begin{cases}
e^{-\frac{1}{x^2}} & x\neq 0\\
0&x=0
\end{cases}$$
$f$ is continuous in $x=0$ and derivable $\infty$ times, but $\forall n \: f^{(n)}(0) = 0$, i.e. series doesn't exists.
\paragraph{Reminder (Taylor polynomial)}
If $f,f^\prime, f^{\prime \prime}, f^{(3)}, \dots, f^{(n)}$ are continuous on $[0,x]$ (or $[x,0]$) and $f^{(n+1)}$ exists on open interval then for polynomial
$$p(n) = \sum_{k=0}^n \frac{f^{(k)}}{k!}x^k$$
$$\exists 0<c<x \: R_n(x) = f(x) - p_n(x) = \frac{f^{(n+1)}(c)}{(n+1)!}x^{n+1}$$
\paragraph{Sentence}
Let $f$ derivable infinite times on $(-r,r)$ and suppose $\exists M \forall n \forall -r<x<r$ such that
$$\left|f^{(n)}(x)\right|\leq M^n$$
then series
$$\sum_{k=0}^\infty \frac{f^{(k)}(0)}{k!}x^k$$
converges and equal to $f(x)$ on $(-r,r)$.
\subparagraph{Proof}
For $|x|<r$
$$R_n(x) = f(x) - p_n(x)$$
$$\left| R_n(x) \right| = \left|\frac{f^{(n+1)}(c)}{(n+1)!}\right|\left|x\right|^{n+1} \leq \left|\frac{1}{(n+1)!}\right|\left|x\right|^{n+1}M^{n+1} \leq \frac{\left(rM\right)^{n+1}}{(n+1)!} \stackrel{n\to \infty}{\to} 0$$
\paragraph{Example}
$$f(x)=cos(x)$$
Taylor series is:
$$1-\frac{x^2}{2}+\frac{x^4}{4!} - \dots = \sum_0^\infty (-1)^n \frac{1}{(2n)!}x^{2n}$$
Does $$cos(x) = \sum_0^\infty (-1)^n \frac{1}{(2n)!}x^{2n}$$
$$f^{(n)} \leq 1$$
Than means that series converge to $\cos x$ and by derivation
$$-\sin x = \sum\frac{(-1)^n}{(2n-1)!}x^{2n-1}$$