\paragraph{Example}
$$\sum_{n=1}^\infty \frac{1}{n!}x^n$$
\subparagraph{Solution}
$$\left|\frac{a_{n+1}}{a_n}\right| = \left| \frac{\frac{1}{(n+1)!}x^{n+1}}{\frac{1}{n!}x^n} \right| = \frac{|x|}{n+1} \stackrel{n\to \infty}{\longrightarrow} = 0 < 1$$
So series converge
\paragraph{Lemma}
$$\sum_{k=1}^n a_kb_k = a_nB_n-\sum_{k=1}^{n-1} B_k\left(a_{k+1}-a_k\right)$$
$$B_k = \sum_{k=1}^n b_k$$
\subparagraph{Proof}
\begin{align*}
\sum_{k=1}^n a_kb_k = \sum_{k=1}^n a_k\left(B_k-B_{k-1}\right) = \sum_{k=1}^n a_kB_k-\sum_{k=1}^{n-1} a_{k+1}B_k = a_nB_n + \sum_{k=1}^{n-1} a_kB_k-\sum_{k=1}^{n-1} a_{k+1}B_k =\\= a_nB_n + \sum_{k=1}^{n-1} a_kB_k - a_{k+1}B_k = a_nB_n - \sum_{k=1}^{n-1} \left(a_{k+1} -a_k \right)B_k
\end{align*}
\paragraph{Sentence}
Let $\left\{ a_n \right\}$ decreasing series such that $a_n \to 0$ and $\left\{ b_n \right\}$ sequence such that $\sup_n \left|\sum_{k=1}^n b_k \right| < \infty$ then $\sum_{n=1}^\infty a_nb_n$ converges.
\subparagraph{Proof}
Denote $M = \sup_n \left|\sum_{k=1}^n b_k \right| $
$$\sum_{k=1}^n a_kb_k = \underbrace{a_nB_n}_{\to 0 (bounded \cdot \to 0)}-\sum_{k=1}^{n-1} B_k\left(a_{k+1}-a_k\right)$$
$$\left|B_k\left(a_{k+1}-a_k\right)\right| \leq M(a_k-a_{k+1})$$
$$\sum_{k=1}^{n-1} M_k\left(a_{k+1}-a_k\right) = M(a_1-a_{k+1}) \to Ma_1$$
From comparison sentence, $B_k\left(a_{k+1}-a_k\right)$ converges absolutely.
\paragraph{Note}
Alternative condition on $\left\{ a_n \right\}$ is that $\sum \big| a_{k+1} - a_k \big|$ converges.
\paragraph{Example}
$b_n = \left( -1 \right)^{n+1}$ fulfills conditions, acquiring Leibniz test.
\paragraph{Example}
$$\sum_{n=1}^{\infty} \frac{\sin nx}{n}$$
\subparagraph{Solution}
$a_n = \frac{1}{n}$, $b_n = \sin nx$.
$\left\{a_n\right\}$ fulfills conditions of the sentence. We need to show that $B_n = \sum_{k=1}^n \sin kx$ is bounded. We can suppose that $\sin \frac{x}{2} \neq 0$, else $x = 2\pi m$ and $B_n = 0$.
$$B_n = \sum_{k=1}^n \sin kx = \sum_{k=1}^n \frac{2\sin \frac{x}{2} \sin kx}{2\sin \frac{x}{2}}$$
$$2\sin \alpha \sin \beta = \cos \left(\alpha - \beta \right) -  \cos \left(\alpha + \beta \right)$$
$$B_n =  \frac{1}{2\sin \frac{x}{2}} \sum_{k=1}^n \left( \cos \left(kx-\frac{1}{2}x\right) - \cos \left(kx+\frac{1}{2}x\right) \right) = \frac{1}{2\sin \frac{x}{2}} \left(\cos \frac{x}{2} - \cos \frac{2n+1}{2}x\right)$$
$$|B_n| \leq \frac{2}{2\left|\sin \frac{x}{2}\right|}$$
\paragraph{Example}
$$\sum \frac{\sin nx}{n+\left(-1\right)^{n+1}}$$ according to alternative condition.
\paragraph{Sentence}
Suppose $\sum^{\infty} b_n$ converges, and $\left\{a_n\right\}$ monotonous and bounded then $\sum a_nb_n$ converges.
\subparagraph{Proof}
Suppose $\left\{a_n\right\}$ is non-decreasing.
$$\sum_{k=1}^n a_kb_k = \underbrace{a_nB_n}_{converges}-\sum_{k=1}^{n-1} B_k\left(a_{k+1}-a_k\right)$$
$$|B_n| \leq M$$
$$\sum_{k=1}^{n-1} \left|B_k\left(a_{k+1}-a_k\right)\right| \leq M(a_{k}-a_{k+1})$$
\paragraph{Example}
$$\sum \underbrace{\frac{\left(-1\right)^{n+1}}{n}}_{b_n} \cdot \underbrace{\left(1+\frac{2}{n}\right)^n}_{a_n}$$
\subsection{Operations on series}
\paragraph{Sentence}
If series $\sum a_n$ converges then the series which is acquired by adding bracket converges too.
\paragraph{Sentence}
If in every bracket all the elements are of same sign and the acquired series converges, then the original series converges too.
\subparagraph{Proof}
Denote $S_n$ paritial sums of $\sum a_n$. Denote the acquired series as $\sum A_n$ and its partial sums as $T_n$.
$$\underbrace{\left(a_1+a_2+\dots+a_{n_1}\right)}_{A_1}+\underbrace{\left(a_{n_1+1}+a_{n_1+2}+\dots+a_{n_2}\right)}_{A_2}+\dots$$
$$T_1 = S_{n_1} \quad T_2 = S_{n_2} \dots$$
Suppose that $\sum^\infty A_k = A$ converges. $$\forall n \exists m n_m+1 \leq n \leq n_{m+1}$$
$$S_n= S_{n_m}+ \sum_{k=n_m+1}^n a_k = T_m + \underbrace{\sum_{k=n_m+1}^n a_k}_{\beta_n}$$
$$\left|\beta_n\right|\leq \left|A_{m+1}\right| \stackrel{m\to \infty}{\to} 0$$
since $A_n$ converges.
$$\beta_m \to 0 \:w T_m \to A \Rightarrow S_n \to A$$
\paragraph{Sentence}
If series absolutely converges then after change of elements order it'll absolutely converge to the same value.
\subparagraph{Proof}
Denote:
$$a_n^+ = \max (a_n,0) \geq 0$$
$$a_n^- = -\min (a_n,0) \geq 0$$
Always $a_n = a_n^+ - a_n^-$ and $|a_n| = a_n^+ + a_n^-$. Also $a_n^\pm \leq |a_n|$. 
Since $\sum a_n$ converges, both $\sum a_n^+$ and $\sum a_n^-$ converges and $$\sum a_n = \sum a_n^+ - \sum a_n^-$$
Change of order is done with permutation $\pi: \mathbb{N} \to \mathbb{N}$. The new series is $\sum a_{\pi_n}$. Both $\sum a_{\pi_n}^+$ and  $\sum a_{\pi_n}^-$ converge to the same value, since they're nonnegative. trivially,
$$\sum a_{\pi_n} = \sum a_{\pi_n}^+ - \sum a_{\pi_n}^- = \sum a_n$$

