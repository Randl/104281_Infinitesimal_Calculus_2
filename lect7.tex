\section{Infinite series}
Series is infinite sum.
$$\sum_{k=1}^\infty a_k = a_1+a_2+\dots$$
For example
$$1+\frac{1}{2}+\frac{1}{4}+\frac{1}{8}+\frac{1}{16}+\frac{1}{32}+\dots$$
Denote $S_n = \sum_{k=1}^{n}$ and define sequence $\left\{ S_n \right\}$.
\paragraph{Definition} If $\{ S_n \}$ converges, then series $\sum_{k=1}^\infty a_k $ converges. If $S_n \longrightarrow S$ we say that $\sum_{k=1}^\infty a_k = S$. Else series diverges.
\paragraph{Example} $$a_n = q^n \quad q\in\mathbb{R}$$
$$\sum_{n=0}^\infty a_n = \sum_{n=0}^\infty q^n = 1+q+q^2+q^3+\dots$$
$$S_n = 1+q+q^2+\dots + q^n = \frac{1-q^n}{1-q} \quad (q\neq1)$$
$\left\{ S_n \right\}$ converges $\iff$ $\left|q\right|<1$ ($q=1 \Rightarrow S_n = n+1 \to \infty$). In this case $$\sum_{n=0}^\infty q^n= \frac{1}{1-q}$$
\paragraph{Example} $$\sum_{k=1}^{\infty} \frac{1}{k(k+1)}$$
$$S_n = \sum_{k=1}^{n} \frac{1}{k(k+1)} = \sum_{k=1}^{n} \frac{1}{k} -  \frac{1}{k+1} = \frac{1}{1} - \frac{1}{n+1}$$
$$\left\{S_n\right\} \to 1 \Rightarrow \sum_{k=1}^{\infty} \frac{1}{k(k+1)} = 1$$
\paragraph{Example} $$\sum_{k=1}^{\infty} \frac{1}{\sqrt{k}}$$
$$S_n = \sum_{k=1}^{n} \frac{1}{\sqrt{k}} = 1+\frac{1}{\sqrt{2}}+\frac{1}{\sqrt{3}}+\frac{1}{\sqrt{4}}+\dots+\frac{1}{\sqrt{n}} \geq n \cdot \frac{1}{\sqrt{n}} =\sqrt{n} \to \infty$$
Series diverges.
\paragraph{Claim} If series converges, $a_n \to 0$.
\subparagraph{Proof} $a_n = S_n - S_{n-1}$
\paragraph{Theorem} If $\sum a_n$ and $\sum b_n$ converge, then also $\sum (a_n+b_n)$ and $\sum c a_n$ converge and:
$$\sum (a_n+b_n) = \sum a_n + \sum b_n$$
$$\sum ca_n = c\sum a_n$$
\subparagraph{Proof} from sequences
\paragraph{Cauchy Theorem} Series $\sum a_n$ converges $\iff$ $\forall \epsilon > 0 \exists N$ such that $\forall m > N$ $\left| \sum_{k=m+1}^{n} a_n\right| < \epsilon$
\subparagraph{Proof} from sequences
\paragraph{Example} $$\sum_{k=1}^\infty \frac{1}{k^2}$$
$$\sum_{k=m+1}^{n}\frac{1}{k^2} \leq \sum_{k=m+1}^{n} \frac{1}{k(k-1)} = \sum_{k=m+1}^{n} \frac{1}{k-1}-\frac{1}{k} = \frac{1}{m} - \frac{1}{n} \leq \frac{1}{m} < \epsilon $$
\paragraph{Example} $$\sum_{k=1}^\infty \frac{1}{k}$$
$$\sum_{k=n+1}^{2n}\frac{1}{k}= \frac{1}{n+1}+\frac{1}{n+2}+\frac{1}{n+3}+\dots + \frac{1}{2n} \geq n \cdot \frac{1}{2n} = \frac{1}{2}$$
i.e. series diverges.
\paragraph{Conclusion} If $\sum \left| a_n \right|$ converges, then also $\sum a_n$ converges
\subparagraph{Proof} $\bigg|\sum_{k=m+1}^{n} a_n\bigg| \leq \bigg|\sum_{k=m+1}^{n}\left| a_n \right| \bigg|$
\subsection{Nonnegative series and comparison theorems} $$\sum_{k=1}^{\infty} a_k \quad a_k \geq 0$$
Series converges $\iff$ $\left\{ S_n\right\}$ converges.
\paragraph{Note} Replacement of finite number of series elements doesn't influence convergence
\paragraph{Comparison theorem 1} If exists $k$ such that $\forall n > k \quad 0 \leq b_n \leq a_n$ and $\sum b_n$ converges then $\sum a_n$ converges
\subparagraph{Proof} $\sum a_n \leq \sum b_n$. If $\sum b_n$ converges $\Rightarrow \left\{ \sum b_n \right\}$ bounded  $\Rightarrow \left\{ \sum a_n \right\}$ bounded $\Rightarrow \sum a_n$ converges.
\paragraph{Example} $$\sum_{n=1}^{\infty} \frac{1}{\sqrt{n(n+1)}}$$
$$\frac{1}{\sqrt{n(n+1)}} \geq \frac{1}{n+1}$$
Since $\sum \frac{1}{n+1}$ diverges, $\frac{1}{\sqrt{n(n+1)}}$ also diverges.
\paragraph{Example} $$\sum_{n=1}^{\infty} \frac{\sin n}{n^2}$$
Since $\frac{\left|\sin n\right|}{n^2} \leq \frac{1}{n^2}$, $\sum \frac{\left|\sin n\right|}{n^2}$ converges and so does  $\sum_{n=1}^{\infty} \frac{\sin n}{n^2}$.
\paragraph{Theorem 2} For $a_n, b_n \geq 0$ and $b_n>0$ from some $m$, then 
\begin{enumerate}
	\item If $$0 < \alpha \leq \frac{a_n}{b_n} \leq \beta < \infty$$, then $a_n$ and $b_n$ converge together.
	\item If $\liminf \frac{a_n}{b_n} > 0$ and $a_n$ converges, then $b_n$ converges
	\item If $\limsup \frac{a_n}{b_n} < \infty$ and $b_n$ converges, then $a_n$ converges
	\item If exists $L = \lim \frac{a_n}{b_n}$ and $0< L < \infty$, then $a_n$ and $b_n$ converge together.
\end{enumerate}
\subparagraph{Proof}
\begin{enumerate}
	\item $a_n \leq \beta b_n$ and $b_n \leq \frac{1}{\alpha} a_n$. If $b_n$ converges, then also $\beta b_n$ converges and by theorem 1. Similarly for $a_n$.
	\item Denote $0 < L = \liminf \frac{a_n}{b_n} \Rightarrow \exists N$ such that $\forall n>N \frac{a_n}{b_n} > \frac{L}{2}$ (if $L=\infty$, put $1$ instead of $\frac{L}{2}$). $\forall n>N a_n >  \frac{L}{2}b_n$, and by theorem 1.
	\item   Denote $\infty > L = \limsup \frac{a_n}{b_n} \Rightarrow \exists N$ such that $\forall n>N \frac{a_n}{b_n} < 2L+1$. $\forall n>N a_n <  \left( 2L+1 \right)b_n$, and by theorem 1.
	\item From 2 and 3.
\end{enumerate}
\paragraph{Example} $$\sum \frac{3n}{2n^2+5n+2}$$
$$\frac{3n}{2n^2+5n+2} \sim \frac{1}{n} $$
$$\frac{\frac{3n}{2n^2+5n+2}}{\frac{1}{n}} \to \frac{3}{2}$$
Which means series diverges
\paragraph{Example} $$\sum \left[ \ln \left( 1 + \frac{1}{n} \right) - \frac{1}{n} \right]$$
By L'Hopital, $$\frac{\ln \left( 1+x \right)-x}{x^2} \to -\frac{1}{2}$$
Denote $a_n = \frac{1}{n} - \ln \left( 1 + \frac{1}{n} \right) \geq 0$
$$\frac{a_n}{n^2} \to \frac{1}{2}$$
Since $\sum \frac{1}{n^2}$ converges, $a_n$ converges, and so does original series.

Back to original series, $$S_n = \sum_{k=1}^{n} \ln \left(1-\frac{1}{k}\right) - \sum_{k=1}^{n} \frac{1}{k} = \sum_{k=1}^{n} \ln \left(\frac{k+1}{k}\right) - \sum_{k=1}^{n} \frac{1}{k} = \sum_{k=1}^{n} \big[ \ln \left(k+1\right) - \ln k\big] - \sum_{k=1}^{n} \frac{1}{k}  = \ln \left(n+1\right) - \sum_{k=1}^{n} \frac{1}{k}  $$
This means $\gamma = \lim_{n\to\infty}\ln n - \sum_{k=1}^{n} \frac{1}{k}$ converges. It's called Eulers's (Euler–Mascheroni) constant.
\paragraph{Theorem 3} If $a_n \cdot b_n \geq 0$ and from some place $\frac{a_{n+1}}{a_n}\leq \frac{b_{n+1}}{b_n}$ then if $b_n$ converges, $a_n$ converges too.
\subparagraph{Proof}
$$a_n = a_1 \cdot \frac{a_2}{a_1} \cdot \frac{a_3}{a_2} \cdot \frac{a_4}{a_3} \cdot \dots \cdot \frac{a_n}{a_{n-1}} \leq a_1 \cdot \frac{b_2}{b_1} \cdot \frac{b_3}{b_2} \cdot \frac{b_4}{b_3} \cdot \dots \cdot \frac{b_n}{b_{n-1}} = \frac{a_1}{b_1} \cdot b_n$$
From theorem 1, $a_n$ converges.
\paragraph{Fraction test} 
\begin{enumerate}
	\item If $\frac{a_{n+1}}{a_n}\leq q < 1$ then $\sum a_n$ converges
	\item If $\frac{a_{n+1}}{a_n}\geq 1$ then $\sum a_n$ diverges
	\item If $\limsup \frac{a_{n+1}}{a_n} < 1$ then $\sum a_n$ converges
	
	If $\liminf \frac{a_{n+1}}{a_n} > 1$ then $\sum a_n$ diverges
	\item If $L = \lim \frac{a_{n+1}}{a_n}$ exists, then if $L>1$ series diverges and if $L<1$ series converges.
\end{enumerate}
\subparagraph{Proof}
\begin{enumerate}
	\item From theorem 3 if $b_n = q^n$
	\item Since $a_n \geq 0$ and increases, $a_n$ doesn't converge to 0.
\end{enumerate}